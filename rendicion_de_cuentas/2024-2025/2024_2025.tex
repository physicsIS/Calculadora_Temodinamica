\documentclass[10pt]{article}
\usepackage[utf8]{inputenc}
\usepackage[spanish]{babel}
\usepackage{geometry}
\geometry{margin=2.5cm}
\usepackage{setspace}
\setstretch{1.2}

\title{Rendición de cuentas --- Proyecto Calculadora Termodinámica (Physics in Silico)}
\author{Club de Física Computacional --- \textit{Physics in Silico (PhiS)}}
\date{\today}

\begin{document}

\maketitle

\section*{Introducción}

El proyecto \textbf{Calculadora Termodinámica} surge como una iniciativa académica dentro de un curso de Ingeniería Mecánica, donde sus creadores, \textbf{Queshia Porras} y \textbf{Hernán Barquero}, desarrollaron una herramienta computacional para el cálculo automatizado de propiedades termodinámicas. 
Tras la finalización del curso, se propuso incorporar el proyecto al repositorio del club \textit{Physics in Silico (PhiS)}, al reconocer su potencial para convertirse en un recurso de apoyo para la enseñanza y la investigación en termodinámica aplicada.

Posteriormente, con el interés de mejorar y extender la funcionalidad del código, \textbf{Hernán Barquero} continuó el desarrollo del proyecto e invitó a \textbf{Barnald Bocker} a unirse al equipo de trabajo. 
Durante este proceso, el proyecto fue presentado en la \textbf{ExpoMeca 2025}, evento en el que obtuvo el \textbf{primer lugar} según la evaluación de un jurado experto, y fue además invitado al \textbf{LXIII MiniCongreso 2025 del CIGEFI}.

\section*{Descripción del proyecto}

La \textit{Calculadora Termodinámica} es una herramienta de software diseñada para calcular propiedades de estado en ciclos termodinámicos, proporcionando una evaluación precisa de cada estado y de los puntos intermedios dentro de los procesos del ciclo.

Las propiedades calculadas incluyen:
\begin{itemize}
    \item Presión
    \item Volumen específico
    \item Temperatura
    \item Energía interna específica
    \item Entalpía específica
    \item Entropía específica
\end{itemize}

El programa permite realizar estos cálculos de manera automatizada según el modelo termodinámico seleccionado. Actualmente, incorpora:
\begin{itemize}
    \item El modelo de \textbf{Gas Ideal} (completamente implementado)
    \item El modelo de \textbf{Van der Waals}, con un nivel de desarrollo aproximado del 70\%
\end{itemize}

Este enfoque modular permitirá en el futuro extender la herramienta hacia otros modelos de ecuaciones de estado y optimizar su rendimiento para aplicaciones en docencia y simulación.

\section*{Miembros del proyecto}

\subsection*{Queshia Porras (Colaboradora externa)}
Cofundadora del proyecto y responsable del diseño conceptual inicial. 
Participó en el desarrollodel proyecto y en la presentación del proyecto en la ExpoMeca 2025. Actualmente no forma parte de este equipo.

\subsection*{Hernán Barquero (Miembro del club)}
Cofundador y actual coordinador del proyecto dentro del club \textit{PhiS}. 
Ha liderado las mejoras estructurales del código y su implementación desde el inicio, la integración de nuevos modelos termodinámicos. 
Participó en la presentación en ExpoMeca 2025 y en la invitación al Minicongreso del CIGEFI.

\subsection*{Barnald Bocker (Miembro del club)}
Se incorporó al equipo tras la finalización del curso original, contribuyendo al desarrollo y optimización del código, integrando el modelo de Van der Waals. 
Ha trabajado en la ampliación de las funciones y en la documentación técnica del proyecto, colaborando en su consolidación como una herramienta computacional estable.
Participó en la presentación en ExpoMeca 2025 y en la invitación al Minicongreso del CIGEFI.

\section*{Conclusión}

Finalmente, la \textit{Calculadora Termodinámica} constituye un ejemplo exitoso de colaboración interdisciplinaria entre estudiantes de física e ingeniería. 
Su integración al club \textit{Physics in Silico (PhiS)} permite su mantenimiento y expansión futura como proyecto de código abierto, orientado a la comunidad académica. 
Este esfuerzo demuestra el compromiso del club con el desarrollo de herramientas científicas accesibles, reproducibles y de alto valor educativo.

\end{document}